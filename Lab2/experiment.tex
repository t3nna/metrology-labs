\section{Experiment}
\subsection{Setup}

\subsection{Measurement methods}

First, we explored different ways of obtaining the parameters' values. We configured the signal generator to $f = \SI{864}{\hertz}$, $V_{pp} = \SI{2.6}{\volt}$ and connected it to the first oscilloscope channel.

\subsubsection*{Direct}

 To obtain parameters' values using the direct method, one must read the peak-to-peak distance from the oscilloscope's display and multiply it by the setting of the sensitivity knob. The picture may be shifted with knobs to facilitate reading.

The vertical sensitivity knob was set to \SI{0.5}{\volt}. Table~\ref{tab:direct-method} shows the parameters' values.

\begin{table}[H]
	\centering
	\begin{tabular}{c|c|c|c|c|c|c}
		$Y [div]$ & $C_{y} [\unit{\volt}]$ & $f [\unit{\hertz}]$ & $V_{pp} [\unit{\volt}]$ & $V_{m} [\unit{\volt}]$ & $V_{0} [\unit{\volt}]$ & $T [\unit{\micro\second}]$\\
		\hline
		5.1 & 0.5 & 864 & 2.55 & 1.275 & 0.3 & 1157
	\end{tabular}
	\caption{Direct method ($Y$ -- peak-to-peak distance, $C_{y}$ -- sensitivity, $f$ -- frequency, $V_{pp}$ -- peak-to-peak voltage, $V_{m}$ -- magnitude, $V_{0}$ -- average (DC) voltage, $T$ -- period)}
	\label{tab:direct-method}
\end{table}

$V_{pp}$, $V_{m}$ and $T$ calculations are shown in Equations~\ref{eq:peak-to-peak},~\ref{eq:magnitude},~\ref{eq:frequency}.


\begin{equation}
	V_{pp} = Y\cdot C_{y} = 5.1\cdot \SI{0.5}{\volt} = \SI{2.55}{\volt}
	\label{eq:peak-to-peak}
\end{equation}

\begin{equation}
	V_{m} = \frac{V_{pp}}{2} = \frac{\SI{2.55}{\volt}}{2} = \SI{1.275}{\volt}
	\label{eq:magnitude}
\end{equation}

\begin{equation}
	T = \frac{1}{f} = \frac{1}{\SI{864}{\hertz}} = \SI{0.001157407407}{\second} = \SI{1157}{\micro\second}
	\label{eq:frequency}
\end{equation}

\subsubsection*{Cursors}

In the cursors method, the measurement is performed indirectly via two pairs of horizontal and vertical cursors. The distance between a pair of cursors is displayed on the screen. Therefore, e.g. signal peak-to-peak voltage may be measured by aligning horizontal cursors with its opposite peaks.

The cursors were positioned at \SI{1.64}{\volt} and \SI{-0.96}{\volt}. Table~\ref{tab:cursors-method} shows the parameters' values.

\begin{table}[H]
	\centering
	\begin{tabular}{c|c|c|c|c|c|c}
		$V_{a} [\unit\volt]$ & $V_{b} [\unit\volt]$ & $f [\unit{\hertz}]$ & $V_{pp} [\unit{\volt}]$ & $V_{m} [\unit{\volt}]$ & $V_{0} [\unit{\volt}]$ & $T [\unit{\micro\second}]$\\
		\hline
		1.64 & -0.96 & 864 & 2.6 & 1.3 & 0.3 & 1157
	\end{tabular}
	\caption{Direct method ($V_{b}$ -- 1st cursor, $V_{b}$ -- 2nd cursor, $f$ -- frequency, $V_{pp}$ -- peak-to-peak voltage, $V_{m}$ -- magnitude, $V_{0}$ -- average (DC) voltage, $T$ -- period)}
	\label{tab:cursors-method}
\end{table}

$V_{m}$ and $T$ calculations are shown in Equations~\ref{eq:magnitude2},~\ref{eq:frequency2}.

\begin{equation}
	V_{m} = \frac{V_{pp}}{2} = \frac{\SI{2.6}{\volt}}{2} = \SI{1.3}{\volt}
	\label{eq:magnitude2}
\end{equation}

\begin{equation}
	T = \frac{1}{f} = \frac{1}{\SI{864}{\hertz}} = \SI{0.001157407407}{\second} = \SI{1157}{\micro\second}
	\label{eq:frequency2}
\end{equation}


\subsubsection*{Measure}

In the measure method, a "measure" button is pressed on the oscilloscope to obtain the results. The peak-to-peak value is then read directly from the display.

Table~\ref{tab:measure-method} shows the parameters' values.

\begin{table}[H]
	\centering
	\begin{tabular}{c|c|c|c|c}
		$f [\unit{\hertz}]$ & $V_{pp} [\unit{\volt}]$ & $V_{m} [\unit{\volt}]$ & $V_{0} [\unit{\volt}]$ & $T [\unit{\micro\second}]$\\
		\hline
		864 & 2.6 & 1.3 & 0.3 & 1157
	\end{tabular}
	\caption{Direct method ($f$ -- frequency, $V_{pp}$ -- peak-to-peak voltage, $V_{m}$ -- magnitude, $V_{0}$ -- average (DC) voltage, $T$ -- period)}
	\label{tab:measure-method}
\end{table}   

$V_{m}$ and $T$ calculations are shown in Equations~\ref{eq:magnitude3},~\ref{eq:frequency3}.

\begin{equation}
	V_{m} = \frac{V_{pp}}{2} = \frac{\SI{2.6}{\volt}}{2} = \SI{1.3}{\volt}
	\label{eq:magnitude3}
\end{equation}

\begin{equation}
	T = \frac{1}{f} = \frac{1}{\SI{864}{\hertz}} = \SI{0.001157407407}{\second} = \SI{1157}{\micro\second}
	\label{eq:frequency3}
\end{equation}

\subsection{Triggering}

In this part of the experiment we observed what happens to displayed signals when different trigger sources are used. We generated two signals and connected them to Channels 1 and 2.

\begin{itemize}
	\item CH1
	\begin{itemize}
		\item $V_{0} = \SI{0}{\volt}$
		\item $V_{pp} = \SI{2.3}{\volt}$
		\item $f = \SI{864}{\hertz}$ or $f = \SI{1}{\kilo\hertz}$
	\end{itemize}
	\item CH2
	\begin{itemize}
		\item $V_{0} = \SI{0}{\volt}$
		\item $V_{pp} = \SI{2}{\volt}$
		\item $f = \SI{4}{\kilo\hertz}$
	\end{itemize}
\end{itemize}

Next, different frequency and trigger combinations were tried. Table~\ref{tab:triggering} shows the collected data.

\begin{table}[H]
	\centering
	\begin{tabular}{c|c|c|c}
		$f_{1} [\unit{\kilo\hertz}]$ & $f_{2} [\unit{\kilo\hertz}]$ & Trigger source & Graph\\
		\hline
		0.864 & 4 & CH1 & Unstable\\
		\hline
		1 & 4 & CH1 & Stable\\
		\hline
		1 & 4 & CH2 & Unstable\\
		\hline
		0.864 & 4 & CH2 & Unstable\\
	\end{tabular}
	\caption{Frequency and trigger combinations}
	\label{tab:triggering}
\end{table}

\subsection{Oscilloscope functions}
In the last part of the experiment we investigated two functions offered by the device: Math and Acquire.

The Math function facilitates performing operations such as signal addition, subtraction, multiplication, etc.

The Acquire function is used to control how the waveform is generated by varying the sample rate of the ADC (analog-to-digital) converter, and offers various acquisition modes. During the laboratory we used the Averaging Acquisition Mode - it averaged out the noise in the taken samples and displayed the underlying signal.
