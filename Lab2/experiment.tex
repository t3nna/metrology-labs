\section{Experiment}
\subsection{Setup}

\subsection{Measurement methods}

First, we explored different ways of obtaining the parameters' values. We set the signal generator to $f = \SI{864}{\hertz}$, $U_{pp} = \SI{2.6}{\volt}$ and connected it to the oscilloscope using Channel 1.

\subsubsection*{Direct}

 To obtain parameters' values using the direct method, one must read the peak-to-peak distance from the oscilloscope's display and multiply it by the setting of the vertical sensitivity knob.

The vertical sensitivity knob was set to \SI{0.5}{\volt}. Table~\ref{tab:direct-method} shows the parameters' values.

\begin{table}[H]
	\centering
	\begin{tabular}{c|c|c|c|c|c|c}
		$Y$ & $C_{y} [\unit{\volt}]$ & $f [\unit{\hertz}]$ & $U_{pp} [\unit{\volt}]$ & $U_{m} [\unit{\volt}]$ & $U_{0} [\unit{\volt}]$ & $T [\unit{\micro\second}]$\\
		\hline
		5.1 & 0.5 & 864 & 2.55 & 1.275 & 0.3 & 1157
	\end{tabular}
	\caption{Direct method ($Y$ -- peak-to-peak distance, $C_{y}$ -- sensitivity, $f$ -- frequency, $U_{pp}$ -- peak-to-peak value, $U_{m}$ -- magnitude, $U_{0}$ -- average (DC) value, $T$ -- period)}
	\label{tab:direct-method}
\end{table}

$U_{pp}$, $U_{m}$ and $T$ calculations are shown in Equations~\ref{eq:peak-to-peak},~\ref{eq:magnitude},~\ref{eq:frequency}.


\begin{equation}
	U_{pp} = Y\cdot C_{y} = 5.1\cdot \SI{0.5}{\volt} = \SI{2.55}{\volt}
	\label{eq:peak-to-peak}
\end{equation}

\begin{equation}
	U_{m} = \frac{U_{pp}}{2} = \frac{\SI{2.55}{\volt}}{2} = \SI{1.275}{\volt}
	\label{eq:magnitude}
\end{equation}

\begin{equation}
	T = \frac{1}{f} = \frac{1}{\SI{864}{\hertz}} = \SI{0.001157407407}{\second} = \SI{1157}{\micro\second}
	\label{eq:frequency}
\end{equation}

\subsubsection*{Cursors}

In the cursors method the picture is first shifted on the display to facilitate reading. The remaining steps are the same as in the direct method.

The cursors were positioned at \SI{1.64}{\volt} and \SI{0.96}{\volt}. Table~\ref{tab:cursors-method} shows the parameters' values. The peak-to-peak value should have been calculated manually but because its value was always shown on the display, we forgot to do it and read it directly from the display.

\begin{table}[H]
	\centering
	\begin{tabular}{c|c|c|c|c|c|c}
		$A [\unit\volt]$ & $B [\unit\volt]$ & $f [\unit{\hertz}]$ & $U_{pp} [\unit{\volt}]$ & $U_{m} [\unit{\volt}]$ & $U_{0} [\unit{\volt}]$ & $T [\unit{\micro\second}]$\\
		\hline
		1.64 & 0.96 & 864 & 2.6 & 1.3 & 0.3 & 1157
	\end{tabular}
	\caption{Direct method ($A$ -- 1st cursor, $B$ -- 2nd cursor, $f$ -- frequency, $U_{pp}$ -- peak-to-peak value, $U_{m}$ -- magnitude, $U_{0}$ -- average (DC) value, $T$ -- period)}
	\label{tab:cursors-method}
\end{table}

$U_{m}$ and $T$ calculations are shown in Equations~\ref{eq:magnitude2},~\ref{eq:frequency2}.

\begin{equation}
	U_{m} = \frac{U_{pp}}{2} = \frac{\SI{2.6}{\volt}}{2} = \SI{1.3}{\volt}
	\label{eq:magnitude2}
\end{equation}

\begin{equation}
	T = \frac{1}{f} = \frac{1}{\SI{864}{\hertz}} = \SI{0.001157407407}{\second} = \SI{1157}{\micro\second}
	\label{eq:frequency2}
\end{equation}


\subsubsection*{Measure}

In the measure method, a "measure" button is pressed on the oscilloscope to obtain the results. The peak-to-peak value is then read directly from the display.

Table~\ref{tab:measure-method} shows the parameters' values.

\begin{table}[H]
	\centering
	\begin{tabular}{c|c|c|c|c}
		$f [\unit{\hertz}]$ & $U_{pp} [\unit{\volt}]$ & $U_{m} [\unit{\volt}]$ & $U_{0} [\unit{\volt}]$ & $T [\unit{\micro\second}]$\\
		\hline
		864 & 2.6 & 1.3 & 0.3 & 1157
	\end{tabular}
	\caption{Direct method ($f$ -- frequency, $U_{pp}$ -- peak-to-peak value, $U_{m}$ -- magnitude, $U_{0}$ -- average (DC) value, $T$ -- period)}
	\label{tab:measure-method}
\end{table}   

$U_{m}$ and $T$ calculations are shown in Equations~\ref{eq:magnitude2},~\ref{eq:frequency2}.

\begin{equation}
	U_{m} = \frac{U_{pp}}{2} = \frac{\SI{2.6}{\volt}}{2} = \SI{1.3}{\volt}
	\label{eq:magnitude2}
\end{equation}

\begin{equation}
	T = \frac{1}{f} = \frac{1}{\SI{864}{\hertz}} = \SI{0.001157407407}{\second} = \SI{1157}{\micro\second}
	\label{eq:frequency2}
\end{equation}

\subsection{Triggering}

In this part of the experiment we observed what happens to signals when different trigger sources are used. We generated two signals and connected them to Channels 1 and 2.

\begin{itemize}
	\item CH1
	\begin{itemize}
		\item $V_{0} = \SI{0}{\volt}$
		\item $V_{pp} = \SI{2.3}{\volt}$
		\item $f = \SI{864}{\hert}$ or $f = \SI{1}{\kilo\hert}$
	\end{itemize}
	\item CH2
	\begin{itemize}
		\item $V_{0} = \SI{0}{\volt}$
		\item $V_{pp} = \SI{2}{\volt}$
		\item $f = \SI{4}{\kilo\hert}$
	\end{itemize}
\end{itemize}

Next, different frequency and trigger combinations were tried. Table~\ref{tab:triggering} shows all collected data.

\begin{table}[H]
	\centering
	\begin{tabular}{c|c|c|c}
		$f_{1} [\unit{\kilo\hertz}]$ & $f_{2} [\unit{\kilo\hertz}]$ & Trigger source & Signal\\
		\hline
		0.864 & 4 & CH1 & Unstable\\
		\hline
		1 & 4 & CH1 & Stable\\
		\hline
		1 & 4 & CH2 & Unstable\\
		\hline
		0.864 & 4 & CH2 & Unstable\\
	\end{tabular}
	\caption{Frequency and trigger combinations}
	\label{tab:triggering}
\end{table}







