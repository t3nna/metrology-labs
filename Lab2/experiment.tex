\section{Experiment}
\subsection{Setup}

In the last lab we used the SDG1025 Generator and "SDS1052DL+" oscilloscope
\subsubsection*{Detailed explanation of how we used it}
SDG1025 Generator
\begin{figure}[H]
	\centering
	\includegraphics[width=12cm]{images/11.png}
	\caption{Schematic example}
	\label{fig:wow1}
\end{figure}

User manual:
\begin{figure}[H]
	\centering
	\includegraphics[width=12cm]{images/12.png}
	\caption{Schematic example}
	\label{fig:wow2}
\end{figure}

- To Set a Waveform we used buttons with waveform icon that which are on \textbf{"waveform keys"} panel.
- With "Sine" button waveform window will display sine waveform.
- By setting frequency/period, amplitude/high level,
offset/low level, sine signal with different parameters can be generated.
- To set freq and another parameters we used \textbf{Number keys}, all parameters displayed in "\textbf{Parameter Area}"
- We used two buttons on the right side of the
   operation panel, which are used to activate or deactivate the output signal.
- There are three sets of buttons on the operation
  panel, which are direction button, the knob and the keypad.
  All of them we used for:
1. The up and down keys were used to shift parameters and the left and right
     keys were used to shift digits.
2. Keypad was used to directly set the parameters value.
3. Knob was used to change a signal digit value whose range is 0~9

\begin{figure}[H]
	\centering
	\includegraphics[width=12cm]{images/13.png}
	\caption{Schematic example}
	\label{fig:wow3}
\end{figure}

Set Sine Signals
\begin{figure}[H]
	\centering
	\includegraphics[width=12cm]{images/14.png}
	\caption{Schematic example}
	\label{fig:wow4}
\end{figure}
\begin{figure}[H]
	\centering
	\includegraphics[width=12cm]{images/15.png}
	\caption{Schematic example}
	\label{fig:wow5}
\end{figure}

 - To Set the Output Frequency/Period we used "\textbf{Sine}" and "\textbf{Freq}" buttons.
   The frequency shown on the screen when the instrument is powered is
   the default value or the set value beforehand. When setting the function,
   if the current value is valid for the new waveform, it will be used
   sequentially. Also we used direction button to select the digit you want to edit and direction button to select the digit you want to edit.
 - We applied the same for the rest of the parameters using "\textbf{Ampl}", "\textbf{Offset}" and "\textbf{Phase}" buttons.

\subsubsection*{"SDS1052DL+" oscilloscope}
\begin{figure}[H]
	\centering
	\includegraphics[width=12cm]{images/16.png}
	\caption{Schematic example}
	\label{fig:wow6}
\end{figure}

\subsubsection*{Menu and Control Button}
- It would be better to start from channel buttons. We used them to turn that channel ON or OFF
  and open the channel menu for that channel. Also we can use the channel menu to
  set up a channel. When the channel is on, the channel button is lit.
- RUN/STOP: Continuously acquires waveforms or stops the acquisition.
- SET TO 50%: We used it to stabilize a waveform quickly. The oscilloscope can
  set the trigger level to be halfway between the minimum and maximum
  voltage level automatically.
- MATH: It used to display the Math menu. We can use the "\textbf{MATH}" menu to use the
  oscilloscopes Math functions.
- "HORI MENU": We used it to display the Horizontal menu. We can use the Horizontal
  menu to display the waveform and zoom in a segment of a waveform.
- MEASURE: Used to display a menu of measurement parameters.
- AUTO: Very useful button. Automatically sets the oscilloscope controls to produce a usable display
  of the input signals.
- SINGLE: Acquire a single waveform and then stops.
- Channel Connector (CH1, CH2): Input connectors for waveforms display.
- The horizontal position control establishes the time between the trigger position
  and the screen center. We can adjust the horizontal “POSITION” knob control to
  view waveform data before the trigger, after the trigger, or some of each. When
  we change the horizontal position of a waveform, we are changing the time
  between the trigger and the center of the display actually
\subsubsection*{Universal Knob}
Deserves a separate group

\begin{figure}[H]
	\centering
	\includegraphics[width=12cm]{images/17.png}
	\caption{Schematic example}
	\label{fig:wow7}
\end{figure}

- Very useful knob. We used the Universal knob with many functions, such as adjusting the
  holdoff time, moving cursors, setting the pulse width, adjusting the upper and lower frequency limit, adjust X and Y masks when
  using the pass/fail function etc. We can also turn the “Universal” knob to
  adjust the storage position of setups, waveforms, pictures when
  saving/recalling and to select menu options.
\subsubsection*{Vertical System}
We used vertical control for displaying waveform, rectify scale and
position.

\begin{figure}[H]
	\centering
	\includegraphics[width=12cm]{images/18.png}
	\caption{Schematic example}
	\label{fig:wow8}
\end{figure}


\subsubsection*{Horizontal System}
As follow Picture, there are one button and two knobs in the HORIZONTAL area.
We used the horizontal controls to change the horizontal scale and position of
waveforms. The horizontal position readout shows the time represented by the
center of the screen, using the time of the trigger as zero. Changing the horizontal
scale causes the waveform to expand or contract around the screen center.

\begin{figure}[H]
	\centering
	\includegraphics[width=12cm]{images/18.png}
	\caption{Schematic example}
	\label{fig:wow9}
\end{figure}


\subsubsection*{Acquiring Signals System}
On the lab we used it last.
When we acquire a signal, the oscilloscope converts it into a digital form and
displays a waveform. The acquisition mode defines how the signal is digitized and
the time base setting affects the time span and level of detail in the acquisition.
- Sampling: In this acquisition mode, the oscilloscope samples the signal in
evenly spaced intervals to construct the waveform. This mode accurately
represents signals most of the time.
- Advantage: We can use this mode to reduce random noise.
  Disadvantage: This mode does not acquire rapid variations in the signal that may occur between samples. This can result in aliasing may cause narrow pulses to be
  missed.

\subsection{Measurement methods}

First, we explored different ways of obtaining the parameters' values. We set the signal generator to $f = \SI{864}{\hertz}$, $U_{pp} = \SI{2.6}{\volt}$ and connected it to the oscilloscope using Channel 1.

\subsubsection*{Direct}

 To obtain parameters' values using the direct method, one must read the peak-to-peak distance from the oscilloscope's display and multiply it by the setting of the vertical sensitivity knob.

The vertical sensitivity knob was set to \SI{0.5}{\volt}. Table~\ref{tab:direct-method} shows the parameters' values.

\begin{table}[H]
	\centering
	\begin{tabular}{c|c|c|c|c|c|c}
		$Y$ & $C_{y} [\unit{\volt}]$ & $f [\unit{\hertz}]$ & $U_{pp} [\unit{\volt}]$ & $U_{m} [\unit{\volt}]$ & $U_{0} [\unit{\volt}]$ & $T [\unit{\micro\second}]$\\
		\hline
		5.1 & 0.5 & 864 & 2.55 & 1.275 & 0.3 & 1157
	\end{tabular}
	\caption{Direct method ($Y$ -- peak-to-peak distance, $C_{y}$ -- sensitivity, $f$ -- frequency, $U_{pp}$ -- peak-to-peak value, $U_{m}$ -- magnitude, $U_{0}$ -- average (DC) value, $T$ -- period)}
	\label{tab:direct-method}
\end{table}

$U_{pp}$, $U_{m}$ and $T$ calculations are shown in Equations~\ref{eq:peak-to-peak},~\ref{eq:magnitude},~\ref{eq:frequency}.


\begin{equation}
	U_{pp} = Y\cdot C_{y} = 5.1\cdot \SI{0.5}{\volt} = \SI{2.55}{\volt}
	\label{eq:peak-to-peak}
\end{equation}

\begin{equation}
	U_{m} = \frac{U_{pp}}{2} = \frac{\SI{2.55}{\volt}}{2} = \SI{1.275}{\volt}
	\label{eq:magnitude}
\end{equation}

\begin{equation}
	T = \frac{1}{f} = \frac{1}{\SI{864}{\hertz}} = \SI{0.001157407407}{\second} = \SI{1157}{\micro\second}
	\label{eq:frequency}
\end{equation}

\subsubsection*{Cursors}

In the cursors method the picture is first shifted on the display to facilitate reading. The remaining steps are the same as in the direct method.

The cursors were positioned at \SI{1.64}{\volt} and \SI{0.96}{\volt}. Table~\ref{tab:cursors-method} shows the parameters' values. The peak-to-peak value should have been calculated manually but because its value was always shown on the display, we forgot to do it and read it directly from the display.

\begin{table}[H]
	\centering
	\begin{tabular}{c|c|c|c|c|c|c}
		$A [\unit\volt]$ & $B [\unit\volt]$ & $f [\unit{\hertz}]$ & $U_{pp} [\unit{\volt}]$ & $U_{m} [\unit{\volt}]$ & $U_{0} [\unit{\volt}]$ & $T [\unit{\micro\second}]$\\
		\hline
		1.64 & 0.96 & 864 & 2.6 & 1.3 & 0.3 & 1157
	\end{tabular}
	\caption{Direct method ($A$ -- 1st cursor, $B$ -- 2nd cursor, $f$ -- frequency, $U_{pp}$ -- peak-to-peak value, $U_{m}$ -- magnitude, $U_{0}$ -- average (DC) value, $T$ -- period)}
	\label{tab:cursors-method}
\end{table}

$U_{m}$ and $T$ calculations are shown in Equations~\ref{eq:magnitude2},~\ref{eq:frequency2}.

\begin{equation}
	U_{m} = \frac{U_{pp}}{2} = \frac{\SI{2.6}{\volt}}{2} = \SI{1.3}{\volt}
	\label{eq:magnitude2}
\end{equation}

\begin{equation}
	T = \frac{1}{f} = \frac{1}{\SI{864}{\hertz}} = \SI{0.001157407407}{\second} = \SI{1157}{\micro\second}
	\label{eq:frequency2}
\end{equation}


\subsubsection*{Measure}

In the measure method, a "measure" button is pressed on the oscilloscope to obtain the results. The peak-to-peak value is then read directly from the display.

Table~\ref{tab:measure-method} shows the parameters' values.

\begin{table}[H]
	\centering
	\begin{tabular}{c|c|c|c|c}
		$f [\unit{\hertz}]$ & $U_{pp} [\unit{\volt}]$ & $U_{m} [\unit{\volt}]$ & $U_{0} [\unit{\volt}]$ & $T [\unit{\micro\second}]$\\
		\hline
		864 & 2.6 & 1.3 & 0.3 & 1157
	\end{tabular}
	\caption{Direct method ($f$ -- frequency, $U_{pp}$ -- peak-to-peak value, $U_{m}$ -- magnitude, $U_{0}$ -- average (DC) value, $T$ -- period)}
	\label{tab:measure-method}
\end{table}   

$U_{m}$ and $T$ calculations are shown in Equations~\ref{eq:magnitude2},~\ref{eq:frequency2}.

\begin{equation}
	U_{m} = \frac{U_{pp}}{2} = \frac{\SI{2.6}{\volt}}{2} = \SI{1.3}{\volt}
	\label{eq:magnitude2}
\end{equation}

\begin{equation}
	T = \frac{1}{f} = \frac{1}{\SI{864}{\hertz}} = \SI{0.001157407407}{\second} = \SI{1157}{\micro\second}
	\label{eq:frequency2}
\end{equation}

\subsection{Triggering}

In this part of the experiment we observed what happens to signals when different trigger sources are used. We generated two signals and connected them to Channels 1 and 2.

\begin{itemize}
	\item CH1
	\begin{itemize}
		\item $V_{0} = \SI{0}{\volt}$
		\item $V_{pp} = \SI{2.3}{\volt}$
		\item $f = \SI{864}{\hert}$ or $f = \SI{1}{\kilo\hert}$
	\end{itemize}
	\item CH2
	\begin{itemize}
		\item $V_{0} = \SI{0}{\volt}$
		\item $V_{pp} = \SI{2}{\volt}$
		\item $f = \SI{4}{\kilo\hert}$
	\end{itemize}
\end{itemize}

Next, different frequency and trigger combinations were tried. Table~\ref{tab:triggering} shows all collected data.

\begin{table}[H]
	\centering
	\begin{tabular}{c|c|c|c}
		$f_{1} [\unit{\kilo\hertz}]$ & $f_{2} [\unit{\kilo\hertz}]$ & Trigger source & Graph\\
		\hline
		0.864 & 4 & CH1 & Unstable\\
		\hline
		1 & 4 & CH1 & Stable\\
		\hline
		1 & 4 & CH2 & Unstable\\
		\hline
		0.864 & 4 & CH2 & nstable\\
	\end{tabular}
	\caption{Frequency and trigger combinations}
	\label{tab:triggering}
\end{table}

\subsection{Oscilloscope functions}
In the last part of the experiment we investigated two functions offered by the device: Math and Acquire.

The Math function facilitates performing operations such as signal addition, subtraction, multiplication, etc.

The Acquire function is used to control how the waveform is generated by varying the sample rate of the ADC (analog-to-digital) converter, and offers various acquisition modes. During the laboratory we used the Averaging Acquisition Mode - it averaged out the noise in the taken samples and displayed the underlying signal.
